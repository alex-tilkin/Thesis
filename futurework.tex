\chapter{Summary, Conclusions, and Future Work}
In the introduction, I expressed the hope that the work in this thesis could be a "first step" towards . In this final chapter, I will conclude by describing the progress made towards this goal in terms of my development of the prototype, the experiments that we made, the concepts that are developed and additional products that yielded during the research. I will also suggest some future research directions that could provide the next steps along the path to a practical and widely applicable inference system.

During the research we explored the domain of software development tools and programming environments. We developed a concept where the programmer may use mobile devices as programming environments in addition to PCs and laptops. We had some question that we had to answer before starting to develop a concept and a prototype for such an idea. What language are we going to use? How code will be represented on small screes? how the programmer will interact with the mobile device?

After we found solutions to all those questions we decided that we need to do a set of experiments to understand how a programmer interacts with a programming environment when s/he needs to use his or her voice. After we did all the experiments we extracted the most significant information in order to understand how programmers tend to describe the code that they want to be written. The information that we extracted is essential for this study because it gave us a good direction how to proceed with the study. 

After we finished with the experiments we decided that we need to explore the existing programming environments and features in different programming languages, especially Java. We collected a huge set of programming features, we will need them in future work to expand the usability of the application that will be built.

Based on the futures that we collected we developed a series of concepts of how code may be represented on small screen (mobile phones and tablets). Those concepts may be used in future work to complete the prototype and to prove that development on mobile devices using voice and touch is possible.

We had an idea allowing every programmer to configure the representation of the code that makes the most sense to him or her. Meaning, every programmer will be able to configure the representation of the code that will be easier for understating. This is very advanced idea and not easy for implementation but we managed to develop a fundamental concept of how it may work.

As mentioned before we found that in order to allow the programmer to code on mobile devices we have to develop an idea of program by voice. Therefore, we researched related works, available system and tools and developed two modules: speech to text engine, based on Google Speech Server, and a context free grammar parser and lexer. We used Antlr4 for this. Those two module will be used in future works.

During the research we gave lecture 3 hours long in a research seminar of Advanced Software Tools in The Blavatnik school of computer science in Tel-Aviv university. And we published a paper in MobileSoft conference which is part of the ICSE conference. The conference took place in Florence in May 2015, we gave a talk in this conference.

For future work we plan to use the modules that have been developed until today and to develop an IDE for mobile devices. In addition, the Parser and Lexer are designed and written in such way so it will be very easy to expand their usability.

All projects are committed to GitHub and they are public, which means every programmer is able to connect to those repository and contribute at hi of her will.