\chapter{Summary}
In this final chapter, I conclude by describing the progress made towards researching the field of programing by using natural language in terms of the prototype development, the experiments that we carried out, the concepts that have been developed, and additional products that were developed during the research. I also suggest some future research directions that could provide the next steps along the path to a practical and widely applicable inference systems.

We have explored the domain of software development tools and programming environments. We have developed a concept enabling the programmer to use mobile devices as programming environments in addition to PCs and laptops. We encountered a number of questions that we had to answer before starting to develop a concept and a prototype for such an idea. What language are we going to use? How will the code be represented on small screens?  How will the programmer interact with the mobile device?

After we found solutions to all those questions we decided that we need to do a set of experiments to understand how a programmer interacts with a programming environment when s/he needs to use his or her voice. Following all the experiments we extracted the most significant information in order to understand how programmers tend to describe the code that they want to be written. The information that we extracted is essential for this study because it allowed us to determine the best direction concerning how to proceed with the study.

After we completed the experiments, we decided that we need to explore the existing programming environments and features in different programming languages, especially Java. We collected a huge set of programming features, which we will need in future work in order to expand the usability of the application that will be built.
Based on the features that we collected, we developed a series of concepts of how code may be represented on a small screen (mobile phones and tablets). Those concepts may be used in future work to complete the prototype and to prove that development on mobile devices using voice and touch is possible.
We had the idea of allowing every programmer to configure the representation of the code that makes the most sense to him or her. Meaning, every programmer would be able to configure the representation of the code that would be easier for understanding. This is a very advanced idea and not easy for implementation, but we managed to develop a fundamental concept of how it may work.

As mentioned before, we found that in order to allow the programmer to code on mobile devices we have to develop an idea of programming by voice. Therefore, we researched related works, available systems and tools and developed two modules: a speech to text engine, based on Google Speech Server, and a context free grammar parser and lexer. We used Antlr4 for this. Those two modules will be used in future works.

During the research process we gave a three hour lecture in a research seminar of Advanced Software Tools in The Blavatnik School of Computer Science in Tel-Aviv University. Also, we published a paper in the MobileSoft Conference which is part of the ICSE conference. The conference, where we gave a presentation, took place in Florence in May 2015.
For future work we plan to use the modules that have been developed up to today and to develop an IDE for mobile devices. In addition, the Parser and Lexer are designed and written in such a way so it will be very easy to expand their usability.

All projects are committed to GitHub and they are public, which means every programmer is able to connect to that repository and contribute as he or she wishes.