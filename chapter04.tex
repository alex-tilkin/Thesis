\chapter{Compact Representation}
\section{Introduction}
Programming languages weren't designed to be presented on small screens, and therefore when you try to present programs on small screens it is difficult to read them. Programming languages are textual, and so require a keyboard In order to edit programs. Standard on-screen keyboards are inconvenient for texting, let alone for programming. In addition, they require one third of the screen which makes the small screen even smaller. Our approach is to use conventional languages such as Java and C++, but allow each programmer to have a tailored compact view that fits a small screen. We believe that we should support programmers better in doing what they already know how to do instead of requiring them to learn a new language and tools only for the purpose of sometimes developing code on mobile phones. Deverywhere is not a new language; it is a way for each programmer to see the code in the way that makes the most sense to him or her. Our solution for small screens is a compact representation of the code, which means focusing on the important information necessary to understand the code. 

This chapter discusses how the code can be presented in compact representation. It covers most of the major domains of programming idioms. Every topic that discussed in this chapter is accompanied with an explanation how it will be configurable by the user and examples of is provided by default.
\section{Operators}
This section discusses how atomic operators will be represented. Every symbol may be modified by the user to any character or icon that s/he wants.
\subsection{Basic Operators}
The following table provides a set of basic atomic operators that used in almost every line of code.

\begin{table}[H]
\centering
\begin{tabular}{|l|l|}
\hline
{\bf Operator} & {\bf Symbol} \\ \hline
Plus & + \\ \hline
Minus & - \\ \hline
Multiplication & * \\ \hline
Division & / \\ \hline
Modulo & \% \\ \hline
Increment & ++ \\ \hline
Decrement & -- \\ \hline
Null & $ \perp $ \\ \hline
Shift & $ \ll $; $ \gg $; $ \ggg $ \\ \hline
Relational & <; >; $ \leq $; $ \geq $ \\ \hline
Equality & =; $ \neq $ \\ \hline
Bitwise AND & \& \\ \hline
Bitwise exclusive OR & $ \textasciicircum $ \\ \hline
Bitwise inclusive OR & | \\ \hline
Logical AND & $ \wedge $  \\ \hline
Logical OR & $ \vee $  \\ \hline
Ternary & ?; : \\ \hline
Assignment & $ \longleftarrow $; +$ \longleftarrow $; -$ \longleftarrow $; *$ \longleftarrow $; /$ \longleftarrow $; \%$ \longleftarrow $; \&$ \longleftarrow $; $ \wedge $$ \longleftarrow $; $ \vee $$ \longleftarrow $; $ \ll $$ \longleftarrow $; $ \gg $$ \longleftarrow $; $ \ggg $$ \longleftarrow $\\ \hline
\end{tabular}
\caption{This table presents operators and symbols that represent them in compact representation. The symbols are examples for how symbols may be presented. The user may modify to any representation that s/he wants. This idea is discussed in \ref{fdsfsd}. Note: ; used as a separator between operator symbols.}
\end{table}