\chapter{Experiments}
\section{Introduction}
This chapter provides information about experiments that have been performed. The main goal of those experiments is to understand how we pronounce the code that we want to insert. The secondary goal is to create a repository of commands that will grouped into categories. Based on the repository will create templates that will help to analyze the pronounced commands.

Every experiment contained of two active participants and two passive participants (passive participants are listeners). One of the active participants was the speaker and the other one was the typer. In every experiment the typer gave to the speaker a programming task where he need to implement a program. 

The speaker had to dictate a program and the typer had to type exactly what the speaker dictated. The speaker had to dictate lines of code in such way so the typer could understand what does he means but not too detailed. For example, if the speaker had to dictate the code in \autoref{fig1}. He would dictate it like this, \textit{"For each element in elements call to to string"}. 

The typer need to follow dictations of the speaker and to type the code to the text editor (all four participants could see the screen). The typer typed the code in Java. Every one of the participants could participate and provide suggestion for pronouncing the commands. The typer could delete, edit and navigate in the code with no limitations. No time constrains and no limitations on the amount of lines. All experiments have been recorded.

After all experiments have been performed we analyzed them and extracted only the relevant lines that represent commands. For each experiment we created a table that contained of two columns. The left column contains the commands that have been dictated and the right column represents the code that has been typed.

\remark{The commands that have been inserted into the tables are filtered from irrelevant vowles. For example, the commands \textit{create class ummm look up} (where \textit{ummm} is the vowel) has been converted to \textit{create class look up}}
\begin{figure}[H]
\begin{lstlisting}
foreach(Element element in elements){
	element.toString();
} 
\end{lstlisting}
\caption{A simple foreach loop where every item in elements activates it's toString method}
\label{fig1}
\end{figure}
\section{Experiment No.1}
\begin{itemize}
	\item Date: 28/Apr/2014.
	\item Speaker: Alex Tilkin.
	\item Typist: Ari Gam.
	\item Description: Implement a small program that contains an interface called \textit{Lookup}. This interface has one method called \textit{find}. It returns \textit{Object} and receives \textit{String}. a class called \textit{SimpleLookup} that implements \textit{Lookup}. It has two private members: \textit{Names} that is an array of \textit{String}s, and \textit{Values} that is an array of \textit{Object}s. The implemented method \textit{find} iterates over all elements in \textit{Names} and compares every one of them with \textit{Name}. If it finds such element it returns the matched element. In addition a method called \textit{processValues} that receives: \textit{String[] names}, and \textit{Lookup table} (the program presented to the speaker during all the experiment).
\end{itemize}
\autoref{tab1} represents the order of the commands that have been dictated (top to bottom). \autoref{fig2} represents the code that was presented to the speaker. \autoref{fig3} represents the results of the dictation.
\begin{figure}[H]
	\begin{lstlisting}
	interface Lookup {
		Object find(String name);
	}
	
	void processValues(String[] names, Lookup table) {
		for (int i = 0; i != names.length; i++) {
			Object value = table.find(names[i]); 
			if (value != null)
				processValue(names[i], value); 
		}
	}
	
	class SimpleLookup implements Lookup {
		private String[] Names;
		private Object[] Values;
		
		public Object find(String name) {
			for (int i = 0; i < Names.length; i++) {
				if (Names[i].equals(name)) 
					return Values[i]; 
				}
				
			return null;
		}
	}
	\end{lstlisting}
	\caption{The original Java code that was presented to the speaker during experiment No. 1}
	\label{fig2}
\end{figure}
\begin{figure}[H]
	\begin{lstlisting}
	interface Lookup{
		Object find(String name){
		}
	}
	
	void processValues(String[] names, Lookup table){
		for(int i = 0; i < names.Length(); i++){
			Object value = table.find(names[i]);
				if(value != null){
					processValues(names[i], value);
				}
			}
		}
		
		class SimpleLookup implements Lookup{
			private Strings[] names;
			private Object[] values;
			
			public Object find(String name){
				for (int i = 0; i < name.Length(); i++){
					if (names[i].equals(name)){
						return values[i];
					}
				}
				
				return null;
			}
		}
	\end{lstlisting}
	\caption{The result of dictating the code in \ref{fig2}}
	\label{fig3}
\end{figure}
\begin{table}[h]
	\begin{tabular}{|p{10cm}|p{6cm}|}
		\hline
		\rowcolor[HTML]{9B9B9B} 
		{\color[HTML]{000000} The speaker said} & {\color[HTML]{000000} The typer typed} \\ \hline
		Create class LookUp & +Class LookUp \\ \hline
		Create a method processValues that returns void and accepts array of strings names and lookupTable & +processValues(names, table) \\ \hline
		Create a loop from zero to the length of names & for 0 $\leq$ i \textless names.length \\ \hline
		Create value type of object accepts table.find, accepts names at i’s index & value $\leftarrow $ table.find(names{[}i{]}) \\ \hline
		If value different from null then & value $\neq$ null ? \\ \hline
		call to processValue that accepts name at i’s index and value & processValue(name{[}i{]}, value) \\ \hline
		We done with processValues &  \\ \hline
		Create a class SimpleLookUp implements LookUp & +Class SimpleLookUP : LookUp \\ \hline
		Delete the last row &  \\ \hline
		Create array of strings call it names and make it private & -{[}{]} names \\ \hline
		Create values type of array of object and make it private & -{[}{]} values \\ \hline
		Create a method that returns an object call it find accepts name type of string and make it public & +find(name) \\ \hline
		Create a loop from zero to the length of names & for 0 $\leq$ i \textless names.length \\ \hline
		If names at i’s index period equals accept name then & names{[}i{]}.equals(name) ? \\ \hline
		Return values at i’s index & $\hookleftarrow$ values{[}i{]} \\ \hline
		Exit the for loop &  \\ \hline
		Return null & null \\ \hline
	\end{tabular}
	\caption{This table presents the major commands that have been dictated during experiment No.1}
	\label{tab1}
\end{table}

\section{Experiment No.2}
\subsection{Part A}
\begin{itemize}
	\item Date: 28/Apr/2014.
	\item Speaker: Ari Gam.
	\item Typist: Alex Tilkin.
	\item Description: Implement the Bubble Sort algorithm . The speaker asked to implement the Bubble Sort algorithm without any assistance. The algorithm had to be implemented in Java.
\end{itemize}
\autoref{tab2} represents the order of the commands that have been dictated (top to bottom). \autoref{fig4} represents the result of the dictation by the speaker in part A.
\begin{figure}[H]
	\begin{lstlisting}
	class BubbleSort{
		public void do(){
			for(int i = 0; i < data.length - 1; i++){
				for(int j = 0; j < i; j++){
					if(data[i] > data[j]){
						int temp = data[j];
						data[j] = data[i];
						data[i] = temp;
					}
				}
			}
		}
		
		private int[] data;
		
		public BubbleSort(int[] init){
			data = new int[init.length];
			for(int i = 0; i < init.length; i++){
				data[i] = init[i];
			}
		}
	}
	\end{lstlisting}
	\caption{The result of the diction of the Bubble Sort algorithm}
	\label{fig4}
\end{figure}
\subsection{Part B}
\begin{itemize}
	\item Date: 28/Apr/2014.
	\item Speaker: Ari Gam.
	\item Typist: Alex Tilkin.
	\item Description: After the speaker has completed the implementation of the Bubble Sort algorithm he has been asked to improve it's time complexity by adding additional condition.
\end{itemize}
\autoref{tab2} represents the order of the commands that have been dictated (top to bottom). \autoref{fig5} represents the result of the dictation by the speaker in part B.
\begin{figure}[H]
	\begin{lstlisting}
	class BubbleSort{
		public void do(){
			
			for(int i = 0; i < data.length - 1; i++){
				boolean done = true;
				for(int j = 0; j < i; j++){
					if(data[i] > data[j]){
						int temp = data[j];
						data[j] = data[i];
						data[i] = temp;
						done = false;
					}
				}
				if(done){
					break;
				}
			}
		}
		
		private int[] data;
		
		public BubbleSort(int[] init){
			data = new int[init.length];
			for(int i = 0; i < init.length; i++){
				data[i] = init[i];
			}
		}
	}
	\end{lstlisting}
	\caption{The result after the additional condition has been  added to the Bubble Sort algorithm}
	\label{fig5}
\end{figure}
\begin{table}[h]
	\begin{tabular}{|p{10cm}|p{6cm}|}
		\hline
		\rowcolor[HTML]{9B9B9B} 
		{\color[HTML]{000000} The speaker said} & {\color[HTML]{000000} The typer typed} \\ \hline
		Create class bubble sort & +Class BubbleSort \\ \hline
		Public void do with no arguments & +Do \\ \hline
		Create array of ints call it data and make it private & -[] data \\ \hline
		Create constructor that receives an array of ints and name it init & +BubbleSort([] init) \\ \hline
		Copy init to data & data = init.clone \\ \hline
		Go to Do method & \\ \hline
		Create a loop from zero to the length of data minus one & for 0 $\leq$ i \textless data.length - 1 \\ \hline
		Create an inner loop from zero to i & for 0 $\leq$ j \textless i \\ \hline
		If the i element of data bigger than the j element of data then switch between them & \begin{tabular}[c]{@{}l@{}}data[i] \textgreater data[j] ?\\ temp $\leftarrow$ data[i]\\ data[i] $\leftarrow$ data[j]\\ data[j] $\leftarrow$ temp\end{tabular} \\ \hline
		\rowcolor[HTML]{9B9B9B}Here starts part B &  \\ \hline
		Go to the beginning of Do &  \\ \hline
		Create a boolean variable done initialized to false & done $\leftarrow$ false \\ \hline
		Add to the exit condition of outer loop not done & for 0 $\leq$ i \textless data.length - 1 \& ~done \\ \hline
		undo & for 0 $\leq$ i \textless data.length - 1 \\ \hline
		Move the statement boolean done initialized to false to the first line of the outer loop & \\ \hline
		Change the value from false to true & done $\leftarrow$ true \\ \hline
		Go to the end of the if & \\ \hline
		Initialize done with false & done $\leftarrow$ false \\ \hline
	\end{tabular}
	\caption{This table presents the major commands that have been dictated during experiment No.2}
	\label{tab2}
\end{table}
\iffalse
%%=========================================
\section{Simple Equations}
Mathematical symbols and equations can written in the text as $\lambda$, $F(t)$, or even $F(t)=\int_0^t \exp(-\lambda x)\,dx$, or as displayed equations
\begin{equation}
F(t)=\int_0^t \exp(-\lambda x)\,dx
\label{eq1}
\end{equation}


The displayed equations are automatically given equation numbers -- here (\ref{eq1}) since this is the first equation in Chapter 2. Note that you can refer to the equation by referring to the ``label'' you specified as part of the equation environment.

You can also include equations without numbers:
\begin{equation*}
F(t)=\sum_{i=1}^n \binom{n}{i}\sin(i\cdot t)
\end{equation*}
\fi
%%=========================================
\subsection*{More Advanced Formulas}
Long formulas that cannot fit into a single line can be written by using the environment \texttt{align} as
\begin{align}
F(t)&= \sum_{i=1}^n \sin(t^{n-1}) - \sum_{i=1}^n \binom{n}{i}\sin(i\cdot t) \\
      & + \int_0^\infty n^{-x} e^{-\lambda x^t}\,dt
\end{align}

In some cases, you need to write ordinary letters inside the equations. You should then use the commands 
\begin{verbatim}
\textrm  and/or \mathrm
\end{verbatim}
The first command returns the normal text font and will be scaled automatically, while the second command will be scaled according to the use.
\begin{equation*}
\textrm{MTTF}= \int_0^\infty R_\mathrm{avg}(t)\,dt
\end{equation*}



Please consult the \LaTeX\ documentation for further details about mathematics in \LaTeX.
%%=========================================
\section*{Definitions}
If you want to include a definition of a term/concept in the text, I have made the following macro (see in \texttt{ramsstyle.sty}):
\begin{defin}
\textbf{Reliability}: The ability of an item to perform a required function under stated environmental and operational conditions and for a stated period of time.\newline
\end{defin}
When text is following directly after the definition, it may sometimes be necessary to end the definition text by the command
\begin{verbatim}
\newline
\end{verbatim}
I have not included this in the definition of the \texttt{defin} environment to avoid too much space when there is not a text-block following the definition.
%%=========================================
\section{Including Figures}
If you use pdf\LaTeX\ (as recommended), all the figures must be in pdf, png, or jpg format. We recommend you to use the pdf format.  Please place the figure files in the directory \textbf{fig}. Figures are included by the command shown for Figure~\ref{fig1}. Please notice the ``path'' to the figure file written by a \emph{forward} slash (/). You should not include the format of the figure file (pdg, png, or jpg) -- just write the ``name'' of the figure. 
\begin{figure}
\centering
\includegraphics[scale=0.6,angle=15]{fig/NTNU}
\caption{This is the logo of NTNU (rotated 15 degrees).}
\label{fig3}
\end{figure}

Each figure should include a unique \emph{label} as shown in the command for Figure~\ref{fig1}. You can then refer to the figure by the \emph{ref} command.
Notice that you can scale the size of the figure by the option \texttt{scale=k}. You may also define a specific width or height of the figure by replacing the \texttt{scale} options by \texttt{width=k} or \texttt{height=k}. The factor \texttt{k} can here be specified in mm, cm, pc, and many other length measures. You may also give \texttt{k} as a fraction of the width of the text or of the height of the text, for example, \texttt{width=0.45$\backslash$textwidth}. If you later change the margins of the text, the figure width will change accordingly. As illustrated in Figure~\ref{fig1}, you may also rotate the figure -- and also do many other things (please check the documentation of the package \texttt{graphicx} -- it is available on your computer, or you may find it on the Internet).

In \LaTeX\ all figures are floating objects and will normally be placed at the top of a page. This is the standard option in all scientific reports. If you insist on placing the figure exactly where you declare the figure, you may include the command \texttt{[h]} (here) immediately after $\backslash$\texttt{begin\{figure\}}. If you will force the figure to be located either at the top or bottom of the page, you may alternatively use  \texttt{[t]} or \texttt{[b]}. For more options, check the documentation.

Large figures may be included as a \emph{sidewaysfigure} as shown in Figure~\ref{fig2}:\footnote{You can use a similar command for large tables.}
\iffalse
\begin{sidewaysfigure}
\centering
\includegraphics[scale=1.8]{fig/NTNU}
\caption{This is the logo of NTNU.}
\label{fig2}
\end{sidewaysfigure}
\fi
%%=========================================
\section{Including Tables}
\LaTeX\ has a lot of different options to include tables. Only one of them is illustrated here.

\begin{table}
	\centering\small
	\caption{The degree of newness of technology.}
	\label{tab10}
		\begin{tabular*}{\textwidth}{@{\extracolsep{\fill}}lccc}
			\toprule
			  &\multicolumn{3}{c}{Level of technology maturity}\\
  \cmidrule{2-4}
			Experience with the		   &  & Limited field history or not & New or \\
              operating  condition  & Proven &  used by company/user & unproven \\
        
			\midrule
			  Previous experience & 1 & 2 & 3 \\
		          No experience by company/user & 2 & 3 & 4 \\
		          No industry experience & 3 & 4 & 4 \\
			\bottomrule
		\end{tabular*}
\end{table}

\begin{remark}
Notice that figure captions (Figure text) shall be located \emph{below} the figure -- and that the caption of tables shall be \emph{above} the table. This is done by placing the $\backslash$\texttt{caption} command beneath the command $\backslash$\texttt{includegraphics} for figures, and above the command $\backslash$\texttt{begin\{tabular*\}} for tables.
\end{remark}
%%=========================================
\section{Copying Figures and Tables}
In some cases, it may be relevant to include figures and tables from from other publications in your report. This can be a direct copy or that you retype the table or redraw the figure. In both cases, you should include a reference to the source in the figure or table caption. The caption might then be written as: \textsl{Figure/Table xx: The caption text is coming here \citep{rausand04}.}

In other cases, you get the idea from a figure or table in a publication, but modify the figure/table to fit your purpose. If the change is significant, your caption should have the following format: \textsl{Figure/Table xx: The caption text is coming here \citep[adapted from][]{rausand04}.}

%%=========================================
\section{References to Figures and Tables}
Remember that all figures and tables shall be referred to and explained/discussed in the text. If a figure/table is not referred to in the text, it shall be deleted from the report.
%%=========================================
\section{A Word About Font-encoding}
When you press a button (or a combination of buttons) on your keyboard, this is represented in your computer according to the \emph{font-encoding} that has been set up. A wide range of font-encodings are available and it may be difficult to choose the ``best'' one. In the template, I have set up a font-encoding called UTF-8 which is a modern and very comprehensive encoding and is expected to be the standard encoding in the future. Before you start using this template, you should open the Preferences ->Editor dialogue in TeXworks (or TeXShop if you use a Mac) and check that encoding UTF-8 has been specified. 

If you use only numbers and letters used in standard English text, it is not very important which encoding you are using, but if you write the Norwegian letters æ, ø, å and accented letters, such as é and ä, you may run into problems if you use different encodings. Please be careful if you cut and paste text from other word-processors or editors into your \LaTeX\ file!

\subsubsection*{Warning}
If you (accidentally) open your file in another editor and this editor is set up with another font-encoding, your non-standard letters will likely come out wrong. If you do this, and detect the error, be sure \emph{not} to save your file in this editor!!

This is not a specific \LaTeX\ problem. You will run into the same problem with all editors and word-processors -- and it is of special importance if you use computers with different platforms (Windows, OSX, Linux).

%%=========================================
\section{Plagiarism}
Plagiarism is defined as ``use, without giving reasonable and appropriate credit to or acknowledging the author or source, of another person's original work, whether such work is made up of code, formulas, ideas, language, research, strategies, writing or other form'', and is a very serious issue in all academic work. You should adhere to the following rules:
\begin{itemize}
\item Give proper references to all the sources you are using as a basis for your work. The references should be give to the original work and not to newer sources that mention the original sources.
\item You may copy paragraphs up to 50 words when you include a proper reference. In doing so, you should place the copied text in inverted commas (i.e., ``Copied text follows \ldots''). Another option is to write the copied text as a quotation, for example:
\begin{quote}
Birnbaum's measure of reliability importance of component $i$ at time $t$ is equal to the probability that the system is in such a state at time $t$ that component $i$ is critical for the system.\newline \mbox{} \hfill \citet{rausand04}
\end{quote}
\end{itemize}