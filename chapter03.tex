\chapter{Supported Features}
\section{Introduction}
This chapter discusses about the research that has been done to collect information about existing technologies for the Java language. The purpose of this research is to collect information about potential technologies that can be integrated in to Deverywhere system and to increase it's functionality. All features have been discussed by the research group whether they have a potential to be integrated or not. The features that presented in this chapter are only those which have been chosen to be integrated. Note that this research is flexible and the list of features might be changed. This part is important for the research because our architecture and prototype are designed in such way so all the mentioned features can be integrated into it.

Every feature that discussed in this chapter is followed by a numerical value which represents the priority of the feature. the range of the numbers is 1-4 where 1 is the highest priority and 4 is the lowest priority. The meaning of priority is how important that feature to this research.

\section{List of Features}
\subsection{Programming by Voice (writing)}
\begin{itemize}
	\item Priority: 1
\end{itemize}
Allow the user to program using his voice.
\begin{itemize}
	\item In case the system stumbles a case of ambiguity it will present options to the user which he could choose the most appropriate solution.
	\item The system should distinguish between when the user dictates to it or speaking to someone (this is very complicated feature to implement so it might be postponed to later works).
\end{itemize}
\subsection{Navigation by Voice}
\begin{itemize}
	\item Priority: 2
\end{itemize}
The user could navigate in the code using his voice.
\subsection{Editing by Voice}
\begin{itemize}
	\item Priority: 3
\end{itemize}
The user could edit the code by using his voice.
\subsection{Compact View Mode}
\begin{itemize}
	\item Priority: 1
\end{itemize}
Provide an easy for understanding, comfortable and compact representation for code. Allow the use of emoticons and other graphical symbols in order to represent language features such as: classes, methods, and variables.
\subsection{Refactoring}
Enable the user to perform refactoring operations on the code with voice commands. Due to it is complicated to support all refactoring features we decided to choose several that will be supported (prioritized list where the first one has the highest priority).
\begin{itemize}
	\item Rename Element (Variable, Rename, Field etc.) - Changing the name into a new one that better reveals its purpose
	\item Constructor Using Fields - Create constructor by selecting a couple of fields and and create a constructor that receives those fields as a formal variables that initialize the local fields
	\item Surround with “try-catch” - Surround a chunk of code with “try-catch” statement
	\item Move Element (Method or Field) - move to a more appropriate Class or source file
	\item Push Down - Move fields from derived class to the base class
	\item Pull Up - Move fields from base class to derived class
	\item Self-Encapsulate Field - force code to access the field with getter and setter methods (should be hidden) (described in details in the section “Getters and Setters Identification”)
	\item Change Method Signature
\end{itemize}