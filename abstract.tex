\addcontentsline{toc}{section}{Abstract}
\section*{Abstract}
In this work we developed an idea of a new approach to programming, where the programmer may use mobile devices for programming in addition to PCs and laptops. In this research we related to two major problems: how to present the code on mobile devices, and how the programmer interacts with the mobile device while s/he programs. Previous works have failed to address this issue because they concentrated on the dictation of the code word by word instead of describing what transpires when using natural language. Our approach is to use natural language for interaction between the mobile device and the programmer. During the research we studied programming environments, performed experiments, built concepts for compact representation of the code, and constructed a prototype in which one can dictate code in natural language and it will understand what is meant. This research is important for the development of programming environments, it may present a solid foundation for future work in this domain both for students and for researchers.
The aim of this final project is to investigate the domain of programming environments, and to develop a concept of a new generation of programming environments. The idea is to use mobile devices such as tablets and smart phones as development tools, in addition to PCs and laptops that are already used today and to integrate all of them into one comprehensive solution.

In order to perform such a study, we had to investigate and learn different programming environments as well as the current state of programming. Secondly, we had to understand what the obstacles in developing a programming environment on mobile platforms are, and to provide a solution to every problem. We found that in order to allow programmers to use mobile platforms as development environments, we have to use voice and touch gestures. In order to learn how programmers use speech as a gesture while programming, we performed a set of experiments. Those experiments gave us a sense of the optimum interaction which can be achieved between a computer and a programmer.

Afterwards, we started to investigate the most dominant and common features that are used today in programming environments. Based on those features, we designed concepts of compact representation of the code so it could fit according to the sizes of mobile devices. Based on the concepts of compact representation of the code, we decided to add the possibility of configuring the representation of the code so the programmer could read to code in the way it makes most sense to him or her.

We investigated the fields of natural language processing and programming languages and built two modules. The first one is a speech to text engine which translates spoken commands to textual commands. The second one is a module that accepts textual commands, validates and resolves them. Those two modules are significant to this study because they are keys of success in this study. They will be used in future work.

This work has been published in an academic article entitled "Deverywhere: Develop Software Everywhere".

Based on the research and its results, we are confident that it is possible to develop a full-scale programming environment that will be used on mobile devices, PCs and laptops. In addition, the programmer will be able to seamlessly switch devices in order to proceed with his or her coding.