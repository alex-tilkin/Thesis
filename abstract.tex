\addcontentsline{toc}{section}{Abstract}
\section*{Abstract}
Alex Tilkin, Computer Science, The Academic College of Tel-Aviv Yaffo.\\
Abstract of Master’s Thesis, Submitted 1 November 2015:
Deverywhere: Develop Software Everywhere - A Template-Based Developing Abstraction.\\

In this work we developed an idea of a new approach of programming, where the programmer may use mobile devices for programming in addition to PCs and laptops. In this research with tackled two major problems: how to present the code on mobile devices, and how programmer will interact with the mobile device while s/he programs. Previous work has failed to address this issue because they contrasted on the dictation of the code word by word instead of describing what you to happen using natural language. Our approach is to use natural language for interaction with the mobile device to programmer. During the research we studied programming environments, performed experiments, built concepts for compact representation of the code, and built a prototype where you can dictate code in natural language and it will understand what you mean. This research is important for the development of programming environments, it may be a solid foundation for future work in this domain for students and researchers.

The aim of this thesis is to investigate the domain of programming environments, and to develop a concept of a new generation of programming environments. The idea is to use mobile devices such as tables and smart phones as a development tools, in addition to PCs and laptops that are already used today and to integrate all of them into a one solution.

In order to perform such a study we had to investigate and learn different programming environments and the current state. Afterwards, we had to understand what are the obstacles in developing a programming environment on mobile platforms and to provide a solution to every problem. 

We found that in order to allow programmers to use mobile platforms as development environments we have to use the voice and touch gestures. In order to learn how programmers use speech as a gesture while programming we performed a set of experiments. Those experiments gave us a sense of how the interaction between a computer and a programmer should be.

Afterwards, we started to investigate the most dominant and common features that are used today in programming environments. Based on those features we designed concepts of compact representation of the code so it could fit to the sizes of mobile devices. Based on the concepts of compact representation of the code we decided to add the possibility of configure the representation of the code so the programmer could read to code the way it makes most sense to him or her.

We investigated the fields of natural language processing and programming languages and built two modules. The first one is a speech to text engine which translates spoken commands to textual commands. The second one is a module that accepts textual commands, validates and resolves them. Those two module are significant to this study because they are keys of success in this study. They will be used in future work.

This work published a paper, the name of the article is "Deverywhere: Develop Software Everywhere".

Based to the research and its results we sure that it is possible to develop a full scale programming environment that will be used on mobile devices, PCs and laptops. In addition, the programmer will be able to seamlessly switch devices in order to proceed with his or her codding.